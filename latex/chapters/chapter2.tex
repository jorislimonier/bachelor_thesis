\chapter{Classical theory of Linear Regression}

To be added
\begin{itemize}
    \item how to get $\hat b$ on page 101.
    \item where the $\chi^2$ distribution comes from in page 101
\end{itemize}

\section{Linear models}

We consider the setting of having a sample of $n$ observations
\[
    (\X_1, \Y_1), \ldots ,(\X_n, \Y_n)
\]
where $X_i \in \mathscr{X} \subseteq \R^p, \; i = 1, \ldots, n$ and $Y_i \in \mathscr{Y} \subseteq \R, \; i = 1, \ldots, n$.

\begin{definition}[The linear model]
    The relationship between an observation $\X_i \in \mathscr{X}$ and its outcome $\Y_i \in \mathscr{Y}$ can be established by a linear model, that is
    \begin{equation}
        i = 1, \ldots, n \qquad \Y_{i} = \sum_{j=1}^{p} \bta_{j} \X_{i}^{(j)} + \eps_{i}
    \end{equation}
    where \(\eps_1, \ldots, \eps_n\) are independent and identically distributed (i.i.d.). Moreover, \(\forall \, i = 1, \ldots, n, \) we have that \( \E [\eps_i] = 0\) and each \(\eps_i\) is independent of all of the \( X_j, \; j=1, \ldots, n \).
\end{definition}

Instead of seeing each observation individually we can deal with all of them together by expressing the linear model in matrix notation

\begin{equation}
    \label{eqn: linear model (matrix)}
    \Y = \X \bta + \eps
\end{equation}

\begin{definition}\
    \begin{enumerate}[label=(\alph*)]
        \item $\X$ is called the \textbf{design matrix}. It has dimension $n \times p$. $\X$ consists of stacking the vectors relative to each observation inside of a matrix
            \[
                X =
                \begin{bmatrix}
                    \text{---} & X_1^T  & \text{---} \\
                                & \vdots &            \\
                    \text{---} & X_n^T  & \text{---} \\
                \end{bmatrix}
            \]
        \item $\bta$ is called the \textbf{parameter vector}. It has dimension $p \times 1$.
        \item $\eps$ is called the \textbf{error vector}. It has dimension $n \times 1$.
        \item $\Y$ is called the \textbf{response vector}. It has dimension $n \times 1$.
    \end{enumerate}
\end{definition}

Without loss of generality, and after centering and scaling if necessary, we can assume that

\[
    \forall j = 1, \ldots, p \quad
    \begin{cases}
        \bar{\Y} = \frac{1}{n} \sum \limits_{i=1}^{n} \Y_{i} = 0                                                   \\
        \sgmhat_{j}^{2} := \frac{1}{n} \sum \limits_{i=1}^{n} \left( \X_{i}^{(j)} - \bar{\X}^{(j)} \right)^{2} = 1 \\
    \end{cases}
\]

\section{The least squares method}

We define the objective function \( S(\bta) \) as follows
\begin{equation}
    \label{eqn: least squares objective function}
    S(\bta) = \sum_{i=1}^{n} \varepsilon_{i}^{2} = \eps^{T} \eps = (\Y-\X \bta)^{T}(\Y-\X \bta)
\end{equation}
which may be rewritten as
\begin{align*}
    S(\bta) & = (\Y - \X \bta)^{T} (\Y - \X \bta)                             \\
            & = \Y^T \Y - \Y^T \X \bta - \bta^T \X^T \Y + \bta^T \X^T \X \bta \\
            & = \Y^T \Y - 2 \bta^T \X^T \Y + \bta^T \X^T \X \bta              \\
\end{align*}
The least squares method aims at finding the vector $\btahat$ minimizing \( S \), that is
\[
    \btahat := \arg \min_{\bta} S(\bta)
\]
We find \( \btahat \) by differentiating \( S \) with respect to $\bta$ and setting the result to 0.
\begin{align}
             & \frac{\partial}{\partial \bta} S(\btahat) = 0 \nonumber                                                                  \\
    \implies & \frac{\partial}{\partial \btahat} \left( \Y^T \Y - 2 \btahat^T \X^T \Y + \btahat^T \X^T \X \btahat \right) = 0 \nonumber \\
    \implies & - 2 \X^T \Y + 2 \X^T \X \btahat = 0 \nonumber                                                                            \\
    \implies & \X^T \X \btahat = \X^T \Y \label{eqn: least-squares normal equations}
\end{align}
where equation \eqref{eqn: least-squares normal equations} is called the least squares normal equations.

If we assume that \( \X^T \X \) is invertible, then \eqref{eqn: least-squares normal equations} yields that our least squares estimator \( \btahat \) is given by
\begin{equation}
    \label{eqn: least squares estimator}
    \btahat = (\X^T \X)^{-1} \X^T \Y
\end{equation}
We are interested in estimating the quality of our prediction. The residuals can help us do that.
\begin{definition}[Residuals]
    For a given set of observations \(\Y\), the \textbf{residuals} (or \textbf{vector of residuals}) is the difference between the prediction of our model and the observed value, that is
    \[
        X (\btahat - \bta) \in \R^n
    \]
\end{definition}

We would like a measure that indicates how far our predictions are from the measurements. We will use the prediction error for for that purpose.

\begin{definition}[Prediction error]
    For a given set of observations \(\Y\), the \textbf{prediction error} is the squared \(\ell^2\)-norm of the difference between the prediction of our model and the observed value. In other words, it is the squared residuals, that is
    \[
        \| X (\btahat - \bta) \|_2^2
    \]
\end{definition}


